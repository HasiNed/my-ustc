\documentclass{ned-article}

\setinfo{
    type    =   {实验报告},
    course  =   {计算机网络},
    title   =   {Internet应用实例分析},
    author  =   {朱云沁},
    id      =   {PB20061372},
    date    =   {2022-10-07},
}

\begin{document}

\maketitle

\tableofcontents

\section{简介}

\subsection{实验目的}

\begin{itemize}[noitemsep]
    \item 认识 HTTP, DNS, FTP 等典型应用层协议.
    \item 学习如何使用 Wireshark, \texttt{tcpdump} 等常见数据包分析工具.
    \item 在实际场景中理解Internet的分层, 接口, 协议的原理与实现, 提升对于计算机网络的认识.
\end{itemize}

\subsection{实验原理}

\subsubsection{应用层协议}

应用层是 TCP/IP 模型的最顶层, 定义了主机程序如何与传输层服务交互以使用网络. 用户为了使用某项特定的网络服务, 需要运行相应的网络应用程序, 执行相应的应用层协议, 在 TCP/IP 协议栈的密切配合下, 与对端进行通信, 从而执行相应的功能. 典型的应用层协议如下:

\begin{itemize}
    \item \textbf{超文本传输协议}(HyperText Transfer Protocol, HTTP)是互联网上应用最为广泛的一种网络传输协议. HTTP协议通过URL来标识服务器上的资源, 客户端向URL发送HTTP请求报文, 服务端根据情况返回响应报文. HTTP/1, HTTP/2使用TCP作为传输协议; HTTP/3使用UDP + QUIC代替TCP/IP连接.
    \item \textbf{超文本传输安全协议}(HyperText Transfer Protocol over Secure Socket Layer, HTTPS)利用标准的HTTP协议进行数据通信, 采用SSL/TLS来提供机密性与完整性服务, 并认证客户端与服务器的合法性. 建立HTTPS连接时, 首先TLS握手, 然后执行标准的HTTP过程. 请求文件的URL, 文件内容, 浏览器表单内容等均将被加密.
    \item \textbf{域名系统}(Domain Name System, DNS)是一项用于网络域名管理的互联网服务, 利用分布式数据库将域名和IP地址相互映射. DNS采用层次化的名字空间, 每个层次都有多个域名. 名字空间的划分为不重叠的区域, 不同区域的信息保存在对应的授权名字服务器. 名字服务器以资源记录的形式来维护本区域内的域名相关信息.
    \item \textbf{文件传输协议}(File Transfer Protocol, FTP)是一个用于在计算机网络上在客户端和服务器之间进行文件传输的应用层协议, 利用两个端口分别传输数据流及控制流. FTP的主动模式要求服务端主动向客户端建立数据连接; 被动模式中, 服务端被动接收客户端的连接建立请求, 从而绕过客户端的防火墙.
\end{itemize}

\subsubsection{网络分析工具}

网络分析工具能够抓取Internet运行过程中的交互数据包, 供专业人员进行分析, 从中提取出有效信息加以利用. 其中, \textbf{Wireshark}是常用的GUI抓包分析软件, \textbf{\texttt{tcpdump}}是常用的命令行抓包工具, 两者均支持针对网络层, 协议, 主机, 网络或端口的过滤.

\subsubsection{常用命令}

\begin{itemize}[noitemsep]
    \item \textbf{\texttt{curl}}: 使用URL语法模拟网络请求.
    \item \textbf{\texttt{dig}}:  查询DNS记录.
    \item \textbf{\texttt{ftp}}:  通过FTP协议进行文件传输.
\end{itemize}

\section{实验过程}

\subsection{TCP}

启动 Wireshark, 将 \texttt{capture filter} 设定为 \texttt{tcp port http}, 选择 \texttt{Wi-Fi: en0} 为接口, 如\figref{fig:wireshark-1}~所示. 开始监听数据包. 在\texttt{zsh}中执行指令\texttt{curl -4 http://www.example.com}, 输出结果见附录~\ref{sec:shell-output}. 从 Wireshark 中找到 TCP 连接建立时三次握手的数据包, 如\figref{fig:tcp-1}, \figref{fig:tcp-2}, \figref{fig:tcp-3}~所示; TCP 连接释放时的三个数据包, 如\figref{fig:tcp-4}, \figref{fig:tcp-5}, \figref{fig:tcp-6}~所示. 观察各数据包, 得到信息如\tabref{tab:tcp}~所示.

\begin{table}[H]
    \centering\small
    \begin{tabular}{|c|clllll|}
        \hline
                                                                                         & \multicolumn{1}{c|}{\textbf{握手包1}}                         & \multicolumn{1}{c|}{\textbf{握手包2}} & \multicolumn{1}{c|}{\textbf{握手包3}} & \multicolumn{1}{c|}{\textbf{接收包1}} & \multicolumn{1}{c|}{\textbf{接收包2}} & \multicolumn{1}{c|}{\textbf{接收包3}} \\ \hline
        \textbf{Seq号}                                                                    & \multicolumn{1}{l|}{0}                                     & \multicolumn{1}{l|}{0}             & \multicolumn{1}{l|}{1}             & \multicolumn{1}{l|}{80}            & \multicolumn{1}{l|}{1592}          & 81                                 \\ \hline
        \textbf{Ack号}                                                                    & \multicolumn{1}{l|}{0}                                     & \multicolumn{1}{l|}{1}             & \multicolumn{1}{l|}{1}             & \multicolumn{1}{l|}{1592}          & \multicolumn{1}{l|}{81}            & 1593                               \\ \hline
        \textbf{Flags}                                                                   & \multicolumn{1}{l|}{SYN}                                   & \multicolumn{1}{l|}{SYN, ACK}      & \multicolumn{1}{l|}{ACK}           & \multicolumn{1}{l|}{FIN, ACK}      & \multicolumn{1}{l|}{FIN, ACK}      & ACK                                \\ \hline
        \textbf{Window}                                                                  & \multicolumn{1}{l|}{65535}                                 & \multicolumn{1}{l|}{65535}         & \multicolumn{1}{l|}{2058}          & \multicolumn{1}{l|}{2048}          & \multicolumn{1}{l|}{128}           & 2048                               \\ \hline
        \multirow{2}{*}{\textbf{\begin{tabular}[c]{@{}c@{}}发送方ip地址\\ 及端口号\end{tabular}}} & \multicolumn{6}{c|}{\multirow{2}{*}{114.214.227.95:55407}}                                                                                                                                                                                          \\
                                                                                         & \multicolumn{6}{c|}{}                                                                                                                                                                                                                               \\ \hline
        \multirow{2}{*}{\textbf{\begin{tabular}[c]{@{}c@{}}接收方ip地址\\ 及端口号\end{tabular}}} & \multicolumn{6}{c|}{\multirow{2}{*}{93.184.216.34:80}}                                                                                                                                                                                              \\
                                                                                         & \multicolumn{6}{c|}{}                                                                                                                                                                                                                               \\ \hline
    \end{tabular}
    \caption{TCP 数据包所含信息}
    \label{tab:tcp}
\end{table}

\begin{figure}[H]
    \centering
    \includegraphics[width=\textwidth]{assets/outputs/wireshark-1.png}
    \caption{TCP 实验中 Wireshark 设置}\label{fig:wireshark-1}
\end{figure}

\begin{figure}[H]
    \centering
    \includegraphics[width=\textwidth]{assets/outputs/tcp-1.png}
    \caption{TCP 握手的第一个数据包}\label{fig:tcp-1}
\end{figure}

\begin{figure}[H]
    \centering
    \includegraphics[width=\textwidth]{assets/outputs/tcp-2.png}
    \caption{TCP 握手的第二个数据包}\label{fig:tcp-2}
\end{figure}

\begin{figure}[H]
    \centering
    \includegraphics[width=\textwidth]{assets/outputs/tcp-3.png}
    \caption{TCP 握手的第三个数据包}\label{fig:tcp-3}
\end{figure}

\begin{figure}[H]
    \centering
    \includegraphics[width=\textwidth]{assets/outputs/tcp-4.png}
    \caption{TCP 连接释放时的第一个数据包}\label{fig:tcp-4}
\end{figure}

\begin{figure}[H]
    \centering
    \includegraphics[width=\textwidth]{assets/outputs/tcp-5.png}
    \caption{TCP 连接释放时的第二个数据包}\label{fig:tcp-5}
\end{figure}

\begin{figure}[H]
    \centering
    \includegraphics[width=\textwidth]{assets/outputs/tcp-6.png}
    \caption{TCP 连接释放时的第三个数据包}\label{fig:tcp-6}
\end{figure}

\subsection{HTTP/HTTPS}

在\texttt{zsh}中执行指令\texttt{curl -v http://www.example.com}, 输出结果见附录~\ref{sec:shell-output}. 用Microsoft Edge访问\texttt{https://www.example.com}, 启动开发者工具, 如\figref{fig:web}~所示.

\begin{figure}[H]
    \centering
    \includegraphics[width=\textwidth]{assets/outputs/web.png}
    \caption{用 Microsoft Edge 访问网址 \texttt{http://www.example.com}}\label{fig:web}
\end{figure}

请求报文理解如下:
\begin{enumerate}[noitemsep]
    \item 请求行: \texttt{GET}为请求方法, \texttt{/} 为相对URL, \texttt{HTTP/1.1}为协议版本.
    \item 首部行:
          \begin{itemize}[noitemsep]
              \item \texttt{Host: www.example.com}表示请求的主机为\texttt{www.example.com}.
              \item \texttt{User-Agent: curl/7.85.0}表示发起请求的应用程序为\texttt{curl}, 版本为\texttt{7.85.0}.
              \item \texttt{Accept: */*}表示客户端可接受任意内容类型.
          \end{itemize}
\end{enumerate}

响应报文理解如下:
\begin{enumerate}[noitemsep]
    \item 状态行: \texttt{HTTP/1.1}为协议版本, \texttt{200}为状态码, \texttt{OK}为状态描述.
    \item 首部行:
          \begin{itemize}[noitemsep]
              \item \texttt{Age: 290980}表示资源在代理服务器中的缓存时间为290980秒.
              \item \texttt{Cache-Control: max-age=604800}表示缓存存储的最大周期为604800秒.
              \item \texttt{Content-Type: text/html; charset=UTF-8}表示内容类型为\texttt{text/html}, \\编码为\texttt{UTF-8}.
              \item \texttt{Date: Fri, 07 Oct 2022 01:54:32 GMT}为报文创建的时间.
              \item \texttt{Etag: "3147526947+ident"}表示资源特定版本的标识符.
              \item \texttt{Expires: Fri, 14 Oct 2022 01:54:32 GMT}为资源过期时间.
              \item \texttt{Last-Modified: Thu, 17 Oct 2019 07:18:26 GMT}为资源最后修改时间.
              \item \texttt{Server: ECS (sec/973B)}为服务器软件相关信息.
              \item \texttt{Vary: Accept-Encoding}表示内容协商使用\texttt{Accept-Encoding}首部选择资源代表.
              \item \texttt{X-Cache: HIT}表示缓存命中.
              \item \texttt{Content-Length: 1256}表示内容长度为1256字节.
          \end{itemize}
    \item 数据:
          \begin{itemize}
              \item \texttt{<!doctype html>}表示文档类型为HTML.
              \item \texttt{<html>}标签包含整个HTML文档. \texttt{<head>}标签包含文档的头部, \texttt{<body>}标签包含文档的主体.
              \item \texttt{<title>}标签表示文档的标题. \texttt{<meta>}标签表示文档元数据. \texttt{charset="utf-8"}表示文档使用UTF-8编码.  \\ \texttt{name="viewport" content="width=device-width, initial-scale=1"}表示文档的视口宽度为设备宽度, 初始缩放比例为1.
              \item \texttt{<style>}标签表示文档的样式, \texttt{type="text/css"}表示类型为层叠样式表.
              \item \texttt{<div>}标签表示一个块级元素, \texttt{<h1>}标签表示一个1级标题, \\ \texttt{<p>}标签表示一个段落, \texttt{<a>}标签表示一个超链接.
          \end{itemize}
\end{enumerate}

启动 Wireshark, 将 \texttt{capture filter} 设定为 \texttt{tcp port http || https}, 选择 \texttt{Wi-Fi: en0} 为接口, 如\figref{fig:wireshark-2}~所示. 开始监听数据包. 在\texttt{zsh}中执行\tabref{tab:http}~中第一条指令, 输出结果见附录~\ref{sec:shell-output}. 从 Wireshark 中找到 HTTP请求报文与响应报文所在数据包, 如\figref{fig:http-1}, \figref{fig:http-2}~所示. 将数据包所含信息填入\tabref{tab:http}.

\begin{figure}[H]
    \centering
    \includegraphics[width=\textwidth]{assets/outputs/wireshark-2.png}
    \caption{HTTP 实验中 Wireshark 设置}\label{fig:wireshark-2}
\end{figure}

\begin{figure}[H]
    \centering
    \includegraphics[width=\textwidth]{assets/outputs/http-1.png}
    \caption{\tabref{tab:http}~中第一条指令HTTP请求报文所在数据包}\label{fig:http-1}
\end{figure}

\begin{figure}[H]
    \centering
    \includegraphics[width=\textwidth]{assets/outputs/http-2.png}
    \caption{\tabref{tab:http}~中第一条指令HTTP响应报文所在数据包}\label{fig:http-2}
\end{figure}

\begin{table}[H]
    \centering\small
    \begin{tabular}{|l|l|l|l|l|}
        \hline
        \multicolumn{1}{|c|}{\textbf{指令}}                                                                                    & \multicolumn{1}{c|}{\textbf{协议版本}} & \multicolumn{1}{c|}{\textbf{方法类型}} & \multicolumn{1}{c|}{\textbf{状态码}} & \multicolumn{1}{c|}{\textbf{内容类型}} \\ \hline
        \texttt{curl -v http://www.example.com}                                                                              & HTTP/1.1                           & GET                                & 200                               & text/html                          \\ \hline
        \texttt{curl -v https://www.example.com}                                                                             & HTTPS (HTTP/2.0, TLS 1.2)          & GET                                & 200                               & text/html                          \\ \hline
        \begin{tabular}[c]{@{}l@{}}\texttt{curl -v -d "user=test"} \\ \texttt{-X POST http://example.com/login}\end{tabular} & HTTP/1.1 (XML 1.0)                 & POST                               & 404                               & text/html                          \\ \hline
    \end{tabular}
    \caption{HTTP数据包所含信息}
    \label{tab:http}
\end{table}

在\texttt{zsh}中执行\tabref{tab:http}~中第二条指令, 输出结果见附录~\ref{sec:shell-output}. 发现数据包被TLS加密, 无法直接查看, 如\figref{fig:https-0}~所示. 为了在Wireshark中解密数据包, 作如下配置:
\begin{itemize}[noitemsep]
    \item \texttt{brew install curl}: 利用 Homebrew 重新安装 cURL
    \item \texttt{export PATH="/opt/homebrew/opt/curl/bin:\$PATH"}: 取代 MacOS 自带的 cURL.
    \item \texttt{export CURL_SSL_BACKEND="openssl"}: 将 cURL 的 SSL 后端设置为 OpenSSL.
    \item \texttt{export SSLKEYLOGFILE="\$HOME/.sslkeylogfile"}: 将 TLS 密钥日志导出到文件中.
    \item 在Wireshark中, 选中\texttt{Preferences -> Protocols -> TLS}, 在\texttt{(Pre)-Master-Secret\\ log filename}一栏填入\texttt{\$HOME/.sslkeylogfile}, 如\figref{fig:tls-config}~所示.
\end{itemize}

\begin{figure}[H]
    \centering
    \includegraphics[width=\textwidth]{assets/outputs/https-0.png}
    \caption{\tabref{tab:http}~中第二条指令被加密的数据包}\label{fig:https-0}
\end{figure}

\begin{figure}[H]
    \centering
    \includegraphics[width=0.7\textwidth]{assets/outputs/tls-config.png}
    \caption{HTTPS 实验中 TLS 配置}\label{fig:tls-config}
\end{figure}

再次执行\tabref{tab:http}~中第二条指令, 发现Wireshark能够将数据包解密, 如\figref{fig:https-1}, \figref{fig:https-2}~所示. 将数据包所含信息填入\tabref{tab:http}.

\begin{figure}[H]
    \centering
    \includegraphics[width=\textwidth]{assets/outputs/https-1.png}
    \caption{\tabref{tab:http}~中第二条指令HTTP请求报文所在数据包之一}\label{fig:https-1}
\end{figure}

\begin{figure}[H]
    \centering
    \includegraphics[width=\textwidth]{assets/outputs/https-2.png}
    \caption{\tabref{tab:http}~中第二条指令HTTP响应报文所在数据包之一}\label{fig:https-2}
\end{figure}

在\texttt{zsh}中执行\tabref{tab:http}~中第三条指令, 输出结果见附录~\ref{sec:shell-output}. 从 Wireshark 中找到数据包, 如\figref{fig:http-3}, \figref{fig:http-4}~所示. 将数据包所含信息填入\tabref{tab:http}.

\begin{figure}[H]
    \centering
    \includegraphics[width=\textwidth]{assets/outputs/http-3.png}
    \caption{\tabref{tab:http}~中第三条指令HTTP请求报文所在数据包}\label{fig:http-3}
\end{figure}

\begin{figure}[H]
    \centering
    \includegraphics[width=\textwidth]{assets/outputs/http-4.png}
    \caption{\tabref{tab:http}~中第三条指令HTTP响应报文所在数据包}\label{fig:http-4}
\end{figure}

\subsection{DNS}

启动 Wireshark, 将 \texttt{capture filter} 设定为 \texttt{port 53}, 选择 \texttt{Wi-Fi: en0} 为接口, 如\figref{fig:wireshark-3}~所示. 开始监听数据包. 在\texttt{zsh}中执行\texttt{sudo dscacheutil "flushcache; sudo killall -HUP mDNSResponder}清除DNS缓存, 然后执行\texttt{curl http://www.example.com}, 输出结果见附录~\ref{sec:shell-output}. 从 Wireshark 中找到对应DNS查询报文所在数据包, 如\figref{fig:dns-1}, \figref{fig:dns-2}~所示; DNS响应报文所在数据包 \figref{fig:dns-3}, \figref{fig:dns-4}~所示. 将数据包所含信息填入\tabref{tab:dns}.

\begin{table}[H]
    \centering\small
    \begin{tabular}{|l|l|}
        \hline
        \multicolumn{1}{|c|}{\textbf{项目}} & \multicolumn{1}{c|}{\textbf{数据}}                                                           \\ \hline
        本机 IP 地址和端口号                      & \begin{tabular}[c]{@{}l@{}}114.214.227.95:53431\\ 114.214.227.95:54966\end{tabular}        \\ \hline
        DNS 服务器 IP 地址和端口号                 & 8.8.8.8:53                                                                                 \\ \hline
        传输层协议类型                           & UDP                                                                                        \\ \hline
        目标服务器 URI                         & http://www.example.com/                                                                    \\ \hline
        目标服务器 IP 地址                       & \begin{tabular}[c]{@{}l@{}}93.184.216.34\\ 2606:2800:220:1:248:1893:25c8:1946\end{tabular} \\ \hline
    \end{tabular}
    \caption{DNS数据包中信息}
    \label{tab:dns}
\end{table}

\begin{figure}[H]
    \centering
    \includegraphics[width=\textwidth]{assets/outputs/wireshark-3.png}
    \caption{DNS 实验中 Wireshark 配置}\label{fig:wireshark-3}
\end{figure}

\begin{figure}[H]
    \centering
    \includegraphics[width=\textwidth]{assets/outputs/dns-1.png}
    \caption{DNS IPv4 地址查询报文所在数据包}\label{fig:dns-1}
\end{figure}

\begin{figure}[H]
    \centering
    \includegraphics[width=\textwidth]{assets/outputs/dns-2.png}
    \caption{DNS IPv6 地址查询报文所在数据包}\label{fig:dns-2}
\end{figure}

\begin{figure}[H]
    \centering
    \includegraphics[width=\textwidth]{assets/outputs/dns-3.png}
    \caption{DNS IPv4 地址响应报文所在数据包}\label{fig:dns-3}
\end{figure}

\begin{figure}[H]
    \centering
    \includegraphics[width=\textwidth]{assets/outputs/dns-4.png}
    \caption{DNS IPv6 地址响应报文所在数据包}\label{fig:dns-4}
\end{figure}

指定 8.8.8.8 为 DNS 服务器, 根据\tabref{tab:dig}~的要求, 使用 \texttt{dig} 查询对应的 DNS 记录, 完整输出结果见附录~\ref{sec:shell-output}. 将缺失的命令和结果填入\tabref{tab:dig}~中.

\begin{table}[H]
    \centering\small
    \begin{tabular}{|l|l|l|}
        \hline
        \multicolumn{1}{|c|}{\textbf{查询目标}}                                   & \multicolumn{1}{c|}{\textbf{命令}}             & \multicolumn{1}{c|}{\textbf{结果}}                                                                 \\ \hline
        \begin{tabular}[c]{@{}l@{}}www.baidu.com\\ IPv4地址\end{tabular}        & \texttt{dig www.baidu.com @8.8.8.8}          & \begin{tabular}[c]{@{}l@{}}182.61.200.6\\ 182.61.200.7\end{tabular}                              \\ \hline
        \begin{tabular}[c]{@{}l@{}}jw.ustc.edu.cn\\ IPv6地址\end{tabular}       & \texttt{dig -t aaaa jw.ustc.edu.cn @8.8.8.8} & 2001:da8:d800:642::248                                                                           \\ \hline
        \begin{tabular}[c]{@{}l@{}}202.38.75.11\\ 域名\end{tabular}             & \texttt{dig -x 202.38.75.11 @8.8.8.8}        & infonet.ustc.edu.cn                                                                              \\ \hline
        \begin{tabular}[c]{@{}l@{}}mail.ustc.edu.cn\\ 邮件交换记录(MX)\end{tabular} & \texttt{dig -t mx mail.ustc.edu.cn @8.8.8.8} & \begin{tabular}[c]{@{}l@{}}smtp1.ustc.edu.cn\\ smtp.ustc.edu.cn\\ smtp2.ustc.edu.cn\end{tabular} \\ \hline
        \begin{tabular}[c]{@{}l@{}}i.ustc.edu.cn\\ CNAME\end{tabular}         & \texttt{dig -t cname i.ustc.edu.cn @8.8.8.8} & revproxy.ustc.edu.cn                                                                             \\ \hline
        \begin{tabular}[c]{@{}l@{}}example.com\\ 域名服务器\end{tabular}           & \texttt{dig -t ns example.com @8.8.8.8}      & \begin{tabular}[c]{@{}l@{}}a.iana.servers.net\\ b.iana.servers.net\end{tabular}                  \\ \hline
    \end{tabular}
    \caption{\texttt{dig}命令执行结果}
    \label{tab:dig}
\end{table}

\clearpage
不带域名, 执行\texttt{dig} 命令, 查询根域名服务器, 结果如\figref{fig:dig-1}~所示.

\begin{figure}[H]
    \centering
    \includegraphics[width=0.7\textwidth]{assets/outputs/dig-1.png}
    \caption{\texttt{dig} 命令查询根域名服务器}\label{fig:dig-1}
\end{figure}

执行\texttt{dig PB20061372.ustc.edu.cn}, 发现无法获得有效的查询结果, 如\figref{fig:dig-2}~所示. \texttt{status} 字段为 \texttt{NXDOMAIN}, 表明域名不存在.

\begin{figure}[H]
    \centering
    \includegraphics[width=0.7\textwidth]{assets/outputs/dig-2.png}
    \caption{\texttt{dig} 命令域名不存在}\label{fig:dig-2}
\end{figure}

\subsection{FTP}

进入\texttt{zsh}终端1, 输入命令 \texttt{sudo tcpdump -vvnn -X host home.ustc.edu.cn}, 开始抓取本机和 home.ustc.edu.cn 之间通信的数据包.

进入\texttt{zsh}终端2, 设定为GB2312编码, 输入命令 \texttt{ftp -4 home.ustc.edu.cn}, 进入ftp的交互界面. 登录后, 依次执行命令 \texttt{ls}, \texttt{passive}, \texttt{ls}, \texttt{passive}, \texttt{quit}, 输出结果见附录~\ref{sec:shell-output}.

\subsubsection{主动模式}

主动模式下, \texttt{ls}命令的通信过程大致为:
\begin{enumerate}[noitemsep]
    \item 客户端请求, 发送 PORT 命令, 告知服务器所开放的数据端口.
    \item 服务器响应, 返回 200 状态码, 客户端应答.
    \item 客户端请求, 发送 LIST 命令.
    \item 服务器响应, 主动向客户端数据端口建立TCP连接.
    \item TCP握手完成, 服务器返回 150 状态码.
    \item 服务器发送文件列表, 在数据通道传输.
    \item 传输完成, 服务器, 客户端先后断开TCP连接.
    \item 服务器返回 226 状态码, 客户端应答.
\end{enumerate}

客户端发送 PORT, 服务器返回200, 如\figref{fig:port-1}~所示. 请求的参数表明, 客户端开放的数据端口为$238 \times 256 + 251 = 61179$, 与\figref{fig:port-2}~中建立连接时的数据端口一致.

\begin{figure}[H]
    \centering
    \includegraphics[width=0.7\textwidth]{assets/outputs/port-1.png}
    \caption{FTP主动模式下, 客户端发送PORT命令}\label{fig:port-1}
\end{figure}

\begin{figure}[H]
    \centering
    \includegraphics[width=0.7\textwidth]{assets/outputs/port-2.png}
    \caption{FTP主动模式下, 客户端开放的数据端口}\label{fig:port-2}
\end{figure}

\subsubsection{被动模式}

被动模式下, \texttt{ls}命令的通信过程大致为:
\begin{enumerate}[noitemsep]
    \item 客户端请求, 发送 PASV 命令.
    \item 服务器响应, 返回 227 状态码, 告知客户端所开放的数据端口.
    \item 客户端应答, 主动向服务器数据端口建立TCP连接.
    \item TCP握手完成, 客户端发送 LIST 命令.
    \item 服务器响应, 返回 150 状态码.
    \item 服务器发送文件列表, 在数据通道传输.
    \item 传输完成, 服务器, 客户端先后断开TCP连接.
    \item 服务器返回 226 状态码, 客户端应答.
\end{enumerate}

客户端发送 PASV, 服务器返回227, 如\figref{fig:pasv-1}~所示. 响应的参数表明, 服务器开放的数据端口为$176 \times 256 + 168 = 45224$, 与\figref{fig:pasv-2}~中传输文件列表时的数据端口一致.

\begin{figure}[H]
    \centering
    \includegraphics[width=0.7\textwidth]{assets/outputs/pasv-1.png}
    \caption{FTP被动模式下, 客户端发送PASV命令}\label{fig:pasv-1}
\end{figure}

\begin{figure}[H]
    \centering
    \includegraphics[width=0.7\textwidth]{assets/outputs/pasv-2.png}
    \caption{FTP被动模式下, 服务器开放的数据端口}\label{fig:pasv-2}
\end{figure}

\section{思考题}

\subsection[第1题]{解释 HTTP 中的幂等是什么意思? GET 操作是幂等的吗? POST 呢?}

一个 HTTP 操作幂等, 是指相同请求被执行一次与连续执行多次的效果相同, 不具有副作用. 在正确的实现下, GET 操作是幂等的, 客户端只从服务器获取数据, 服务器的数据不发生改变;  POST 操作不是幂等的, 客户端向服务器提交表单, 服务器处理后, 数据可能发生改变.

\subsection[第2题]{HTTPS 抓到的数据包与之前 HTTP 中抓到的有何不同?这是什么原因导致的?}

\subsubsection*{不同:}
HTTPS实验抓到的数据包被加密, 无法直接查看报文内容. 数据包数量比HTTP实验中抓到的包更多. 解密后, 数据包的格式与内容与HTTP实验中抓到的数据包存在较大差异.

在Wireshark中, HTTP实验中的数据包协议显示为TCP与HTTP, 如\figref{fig:http-1}~所示; HTTPS实验中, 未经解密的数据包协议显示为TLSv1.3, 如\figref{fig:https-0}~所示, 解密后的数据包协议显示为TCP与HTTP2, 如\figref{fig:https-1}~所示.

\subsubsection*{原因:}
HTTPS建立在标准HTTP协议之上, 但采用SSL/TLS来提供机密性, 数据完整性, 服务器鉴别, 客户端鉴别等服务来强化TCP传输. 初始设置下, 使用Wireshark抓包时无法获知TLS密钥, 因此无法解析被TLS加密的数据包内容. 

由于HTTP/2的使用以及TLS的握手, HTTPS实验中产生的数据包更多, 并且传输过程, 报文格式与HTTP/1.1不完全相同.

\subsection[第3题]{FTP 实验中使用的 \texttt{tcpdump} 指令整体可以达到什么效果? 每个参数的含义分别是什么?}

\subsubsection*{效果}
作为命令行工具, \texttt{tcpdump} 将网络上传送的数据包完全捕获, 针对网络层, 协议, 主机, 网络, 端口等条件对数据包进行过滤. 对于与表达式匹配的数据包, 输出其时间戳, 源地址, 目的地址, 协议, 长度, 数据内容等信息. 直至被SIGINT,SIGTERM 等信号中断, 或已处理指定数量的数据包, 输出接受和处理的数据包数量. \texttt{tcpdump} 需要在特权下运行.

\subsubsection*{参数}

\texttt{-v} : 生成详细输出. 例如, 打印 IP 数据包中的生存时间, 标识, 总长度, 选项等. 还支持其他数据包完整性检查, 例如验证 IP 和 ICMP 标头校验和.

\texttt{-vv} : 在\texttt{-v}的基础上, 生成更加详细的输出. 例如, 从 NFS 应答数据包打印其他字段, 并完全解码 SMB 数据包.

\texttt{-n} : 
不将主机地址转换为主机名. 可用于避免DNS查询.

\texttt{-nn} : 
在\texttt{-v}的基础上, 不将协议, 端口号转换为名称. 

\texttt{-X} : 在打印每个数据包标头的基础上, 以十六进制和 ASCII 格式分别打印每个数据包的数据 (除去其链路级标头).

\texttt{host home.ustc.edu.cn} : 过滤数据包的逻辑表达式, 服从\texttt{pcap-filter}语法. 本例中, 当源主机或目的主机为\texttt{home.ustc.edu.cn}时, 表达式为真. 其余参数可查阅man page获知.

\subsection[第4题]{解释从输入网址, 到浏览器显示网页, 在应用层依次发生了什么?}

\begin{enumerate}
    \item \textbf{URL 输入}: 用户在浏览器地址栏输入某个网页的URL. 与此同时, 浏览器在历史记录, 书签等数据中查找当前输入, 给出智能提示. 用户确认后, 浏览器检查输入的内容是否是合法的 URL. 若是, 判断输入的 URL 是否完整, 浏览器补全前缀或者后缀; 若否, 将输入内容作为条件, 使用搜索引擎来进行搜索.
    \item \textbf{DNS 解析}: 根据地址栏输入的域名, 操作系统检查浏览器缓存和本地的 hosts 文件, 若存在对应IP地址的资源记录, 则完成域名解析; 否则, 查询本地域名服务器; 若仍不存在, 经由根域名服务器, 顶级域服务器进行迭代查询, 直到权威域名服务器给出解析结果. 若仍查询失败, 浏览器提示用户无法访问该网站.
    \item \textbf{HTTP/HTTPS 请求}: 假定使用 HTTP/1 或 HTTP/2 进行网页传输. 获取到服务器的 IP 地址后, 浏览器向服务器发起 TCP 连接请求, 双方经过三次握手建立连接. 如果使用 HTTPS, 还需进行TLS握手, 增强通信的安全性. 连接建立后, 浏览器向服务器发送一个初始的 HTTP 请求, 通常为 GET, POST等方法.
    \item \textbf{服务器处理请求}: 服务器接收并解析请求报文, 进行处理, 构建响应报文. 例如: POST 方法需要包含请求实体, 判断是否有访问权限, 指定路径是否存在, 返回缓存还是原始资源. 若为静态资源(HTML, 图片, CSS 等)直接从文件系统获取并返回; 若为动态资源, 需调用 CGI 程序(JavaScript, Python, Rust 等), 返回输出结果; 若配置了负载均衡, 需将请求进行转发.
    \item \textbf{HTTP/HTTPS 响应}: 服务器处理完成后, 将响应报文发送给浏览器. 
    \item \textbf{浏览器渲染页面}: 假定内容类型为 HTML. 浏览器接收到初始响应报文后, 解析 HTML 和 CSS, 构造 DOM 树和 CSSOM 树. 同时, JavaScript 与 WebAssembly 被编译和执行. CSSOM 树和 DOM 树组合为渲染树,用于计算每个可见元素的布局,然后绘制到屏幕上. 若通过HTTP继续请求图像等资源, 渲染过程可能将返回到布局步骤并重新开始. 
    \item \textbf{请求结束, 断开连接}: 浏览器接收到所有响应报文后, 断开 TCP 连接.
\end{enumerate}

\setupappendix

\includepdf[pages=-,addtotoc={
            1,section,1,\texttt{zsh}输出,sec:shell-output,
            1,subsection,1,TCP,sec:tcp
        }]{assets/outputs/tcp.pdf}
\includepdf[pages=-,addtotoc={
            1,subsection,1,HTTP/HTTPS,sec:http,
            1,subsubsection,1,命令1,sec:http-1
        }]{assets/outputs/http.pdf}
\includepdf[pages=-,addtotoc={
            1,subsubsection,1,命令2,sec:http-2
        }]{assets/outputs/https.pdf}
\includepdf[pages=-,addtotoc={
            1,subsubsection,1,命令3,sec:http-3
        }]{assets/outputs/404.pdf}
\includepdf[pages=-,addtotoc={
            1,subsection,1,DNS,sec:dns
        }]{assets/outputs/dns.pdf}
\includepdf[pages=-,addtotoc={
            1,subsection,1,FTP,sec:ftp,
            1,subsubsection,1,终端1,sec:ftp-1
        }]{assets/outputs/tcpdump.pdf}
\includepdf[pages=-,addtotoc={
            1,subsubsection,1,终端2,sec:ftp-2
        }]{assets/outputs/ftp.pdf}

\end{document}