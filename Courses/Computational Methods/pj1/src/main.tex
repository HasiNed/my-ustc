% !TeX encoding = UTF-8
% !TeX program = xelatex
% !TeX spellcheck = en_US

\documentclass[degree=report, output=electronic]{ustcthesis}

% !TeX root = ./main.tex

\ustcsetup{
  title              = {中国科学技术大学\\学位论文模板示例文档 \ustcthesisversion},
  title*             = {An example of thesis template for University of Science
      and Technology of China \ustcthesisversion},
  % course             = {计算方法B},
  % course*            = {Computational Methods B},
  author             = {},
  author*            = {},
  speciality         = {人工智能},
  speciality*        = {Artificial Intelligence},
  supervisor         = {},
  supervisor*        = {},
  % date               = {2017-05-01},  % 默认为今日
  % professional-type  = {专业学位类型},
  % professional-type* = {Professional degree type},
  % department         = {信息科学技术学院},  % 院系,本科生需要填写
  % student-id         = {PB11001000},  % 学号,本科生需要填写
  % secret-level       = {秘密},     % 绝密|机密|秘密|控阅,注释本行则公开
  % secret-level*      = {Secret},  % Top secret | Highly secret | Secret
  % secret-year        = {10},      % 保密/控阅期限
  %
  % 数学字体
  % math-style         = GB,  % 可选:GB, TeX, ISO
  math-font          = xits,  % 可选:stix, xits, libertinus
}


% 加载宏包

% 定理类环境宏包
\usepackage{amsthm}

% 插图
\usepackage{graphicx}

% 三线表
\usepackage{booktabs}

% 跨页表格
\usepackage{longtable}

% 算法
\usepackage[ruled,linesnumbered]{algorithm2e}

% SI 量和单位
\usepackage{siunitx}

% 参考文献使用 BibTeX + natbib 宏包
% 顺序编码制
\usepackage[sort]{natbib}
\bibliographystyle{ustcthesis-numerical}

% 著者-出版年制
% \usepackage{natbib}
% \bibliographystyle{ustcthesis-authoryear}

% 本科生参考文献的著录格式
% \usepackage[sort]{natbib}
% \bibliographystyle{ustcthesis-bachelor}

% 参考文献使用 BibLaTeX 宏包
% \usepackage[style=ustcthesis-numeric]{biblatex}
% \usepackage[bibstyle=ustcthesis-numeric,citestyle=ustcthesis-inline]{biblatex}
% \usepackage[style=ustcthesis-authoryear]{biblatex}
% \usepackage[style=ustcthesis-bachelor]{biblatex}
% 声明 BibLaTeX 的数据库
% \addbibresource{bib/ustc.bib}

% 配置图片的默认目录
\graphicspath{{figures/}}

% 数学命令
\makeatletter
\newcommand\dif{%  % 微分符号
  \mathop{}\!%
  \ifustc@math@style@TeX
    d%
  \else
    \mathrm{d}%
  \fi
}
\makeatother
\newcommand\eu{{\symup{e}}}
\newcommand\iu{{\symup{i}}}

% 用于写文档的命令
\DeclareRobustCommand\cs[1]{\texttt{\char`\\#1}}
\DeclareRobustCommand\pkg{\textsf}
\DeclareRobustCommand\file{\nolinkurl}

% hyperref 宏包在最后调用
\usepackage{hyperref}


\ustcsetup{
  course             = {计算方法B},
  title              = {Project 1},
  author             = {朱云沁},
  student-id         = {PB20061372},
  math-font          = stix,
}

\usepackage[inline]{enumitem}
\usepackage[skins,minted,breakable]{tcolorbox}
\usepackage{xcolor}
\usepackage{float}

\newtcbinputlisting{\pylistings}[2][]{
  skin=empty, breakable, center, listing only,
  listing file={#2},
  minted language=python,
  minted options={frame=lines, framesep=2mm, fontsize=\footnotesize, autogobble, linenos, breaklines}, #1
}

\begin{document}

\maketitle

\frontmatter

\tableofcontents

\mainmatter

\newtheorem{problem}{问题}

\section{实验题目}

\begin{problem}
下面给出美国\numrange{1920}{1970}年的人口表:

\begin{table}[h]
  \centering
  \caption{美国\numrange{1920}{1970}年的人口表}
  \label{tab:ori-data}
  \begin{tabular}{c*{6}{c}}
    \toprule
    年份         & 1920   & 1930   & 1940   & 1950   & 1960   & 1970   \\
    \midrule
    人口(千人) & 105711 & 123203 & 131669 & 150697 & 179323 & 203212 \\
    \bottomrule
  \end{tabular}
\end{table}

用表~\ref{tab:ori-data}~中数据构造一个5次Lagrange插值多项式,并用此估计1910、1965和2002年的人口。1910年的实际人口数约为91772000,请判断插值计算得到的1965年和2002年的人口数据准确性是多少?
\end{problem}

\begin{problem}
数据同表~\ref{tab:ori-data},用Newton插值估计:

\begin{enumerate}[align=left,label=(\arabic*)]
  \item 1965年的人口数;
  \item 2012年的人口数。
\end{enumerate}

\end{problem}

\section{实验结果}

设$x$为年份,$f(x)$为人口(千人),即被插函数。将表~\ref{tab:ori-data}~中年份从左至右分别记作$x_0,x_1,x_2,x_3,x_4,x_5$,并令$x_6=1910$。

\subsection{Lagrange插值}

编写程序,以$x_0,x_1,x_2,x_3,x_4,x_5$为一组插值节点,构造Lagrange插值多项式$L_5(x)$;以$x_1,x_2,x_3,x_4,x_5,x_6$为另一组插值节点,构造Lagrange插值多项式$\tilde{L}_5(x)$。插值计算得1910、1965、2002和2012年的人口,用事后估计方法计算得误差$f(x)-L_5(x)$。结果如表~\ref{tab:lagrange-output}~所示。

\begin{table}[h]
  \centering
  \caption{Lagrange插值数值结果}
  \label{tab:lagrange-output}
  \begin{tabular}{c*{3}c}
    \toprule
    $x$  & $L_5(x)$       & $\tilde{L}_5(x)$ & $f(x)-L_5(x)$ \\
    \midrule
    1910 & 31872.000000   & 91772.000000     & 5.990000e+04  \\
    1965 & 193081.511719  & 193354.493490    & -1.228418e+03 \\
    2002 & 26138.748416   & -233411.305984   & 2.128310e+06  \\
    2012 & -136453.125184 & -801550.139584   & 6.118893e+06  \\
    \bottomrule
  \end{tabular}
  \note{注:数据保留原始输出格式,下同}
\end{table}

绘制得区间$[1900,2010]$上$L_5(x)$、$\tilde{L}_5(x)$以及$f(x)-L_5(x)$的图象如图~\ref{fig:lagrange}~所示。

\newpage

\begin{figure}[h]
  \centering
  \includegraphics[width=\textwidth]{lagrange-interpolation.pdf}
  \caption{Lagrange插值}
  \label{fig:lagrange}
\end{figure}

\subsection{Newton插值}

编写程序,以$x_0,x_1,x_2,x_3,x_4,x_5$为插值节点,构造牛顿插值多项式$N_5(x)$,插值计算得1910、1965、2002和2012年的人口。结果如表~\ref{tab:newton-output}~所示。

\begin{table}[h]
  \centering
  \caption{Newton插值数值结果}
  \label{tab:newton-output}
  \begin{tabular}{cc}
    \toprule
    $x$  & $N_5(x)$           \\
    \midrule
    1910 & 31872.000000000015 \\
    1965 & 193081.51171875    \\
    2002 & 26138.748416000046 \\
    2012 & -136453.1251840004 \\
    \bottomrule
  \end{tabular}
\end{table}

绘制得区间$[1900,2010]$上$N_5(x)$的图象如图~\ref{fig:newton}~所示。

\begin{figure}[h]
  \centering
  \includegraphics[width=\textwidth]{newton-interpolation.pdf}
  \caption{Newton插值}
  \label{fig:newton}
\end{figure}

\subsection{Neville算法}

编写程序,以$x_0,x_1,x_2,x_3,x_4,x_5$为插值节点,用Neville算法构造插值多项式$P_{0,1,\dots,5}(x)$,插值计算得1910、1965、2002和2012年的人口。结果如表~\ref{tab:neville-output}~所示。

\begin{table}[h]
  \centering
  \caption{Neville算法插值数值结果}
  \label{tab:neville-output}
  \begin{tabular}{cc}
    \toprule
    $x$  & $P_{0,1,\dots,5}(x)$ \\
    \midrule
    1910 & 31872.0              \\
    1965 & 193081.51171875      \\
    2002 & 26138.748415998853   \\
    2012 & -136453.12518400417  \\
    \bottomrule
  \end{tabular}
\end{table}

绘制得区间$[1900,2010]$上$P_{0,1,\dots,5}(x)$的图象如图~\ref{fig:neville}~所示。

\begin{figure}[h]
  \centering
  \includegraphics[width=\textwidth]{neville-interpolation.pdf}
  \caption{Neville算法}
  \label{fig:neville}
\end{figure}

\section{算法分析}

\subsection{Lagrange插值}

关于插值节点$x_0,x_1,\dots,x_n$的$n$次Lagrange插值多项式为

\begin{equation}
  L_n(x)=\sum_{i=0}^{n}l_i(x)f(x_i)
\end{equation}

其中,插值基函数$l_i(x)$的表达式为

\begin{equation}
  l_i(x)=\prod_{\begin{subarray}{c}0\le j\le n\\j\neq i\end{subarray}}\frac{x-x_j}{x_i-x_j}
\end{equation}

于是,立即得到计算Lagrange多项式的算法:

\begin{algorithm}[h]
  \SetAlgoLined
  \KwData{$(x_i,f(x_i)),i=0,1,\dots,n$;待计算的点$x$。}
  \KwResult{$L_n(x)$}
  
  $L_n(x)\leftarrow 0$\;
  \For{$i\leftarrow 0$ \KwTo $n$}{
    $l_i(x)\leftarrow 1$\;
    \For{$j\leftarrow 0$ \KwTo $n$}{
      $l_i(x)\leftarrow l_i(x)\cdot (x-x_j)/(x_i-x_j)$\tcp*{对于给定$x$,计算$l_i(x)$}
    }
    $L_n(x)\leftarrow L_n(x) + l_i(x)\cdot f(x_i)$\tcp*{求和,计算$L_n(x)$}
  }
  \caption{Lagrange插值}
  \label{algo:lagrange}
\end{algorithm}

显然,该算法的时间复杂度为$\mathbb{O}(n^2)$,空间复杂度为$\mathbb{O}(1)$。

以$(x_i,f(x_i)),i=1,2,\dots,n+1$为插值点,重复上述算法,得$\tilde{L}_n(x)$,进而用事后估计方法估算得误差为
\begin{equation}
  f(x)-L_n(x)\approx \frac{x-x_0}{x_0-x_{n+1}}(L_n(x)-\tilde{L}_n(x))
\end{equation}

\newpage

\subsection{Newton插值}

关于插值节点$x_0,x_1,\dots,x_n$的$n$次Newton插值多项式为

\begin{equation}
  N(x)=f(x_0)+\sum_{k=1}^{n}f[x_0,x_1,\dots,x_k]\prod_{i=0}^{k-1}(x-x_i)
\end{equation}

其中,$f[x_0,x_1,\dots,x_k]$为函数$f(x)$关于$x_0,x_1,\dots,x_k$的$k$阶差商,满足如下递归关系

\begin{equation}
  f[x_0,x_1,\dots,x_k]=\frac{f[x_1,\dots,x_k]-f[x_0,x_1,\dots,x_{k-1}]}{x_k-x_0}
\end{equation}

构造如下差商表:

\begin{table}[h]
  \centering
  \caption{差商表}
  \label{tab:diff-table}
  \begin{tabular}{c*{5}c}
    \toprule
    $f(x_i)$ & $f[x_{i-1},x_i]$ & $f[x_{i-2},x_{i-1},x_{i}]$ & $\cdots$ & $f[x_0,x_1,\dots,x_i]$ \\
    \midrule
    $f(x_0)$                                                                                     \\
    $f(x_1)$ & $f[x_0,x_1]$                                                                      \\
    $f(x_2)$ & $f[x_1,x_2]$     & $f[x_0,x_1,x_2]$                                               \\
    $\vdots$ & $\vdots$         & $\vdots$                   & $\ddots$                          \\ 
    $f(x_n)$ & $f[x_{n-1},x_n]$ & $f[x_{n-2},x_{n-1},x_n]$   & $\cdots$ & $f[x_0,x_1,\dots,x_n]$ \\
    \bottomrule
  \end{tabular}
\end{table}

观察发现,按从左到右、从下到上的顺序计算上述差商表,只需维护大小为$n+1$的数组,记作$g[i],i=0,1,\dots,n$。递推结束后,应有$g[k]=f[x_0,x_1,\dots,x_k]$,进而计算$N_n(x)$。有如下算法:

\begin{algorithm}[h]
  \SetAlgoLined
  \KwData{$(x_i,f(x_i)),i=0,1,\dots,n$;待计算的点$x$。}
  \KwResult{$N_n(x)$}
  
  \tcp{递推计算$g[k]$,即$f[x_0,x_1,\dots,x_k]$}
  $g[i]\leftarrow f(x_i), i=0,1,\dots,n$\;
  \For{$k\leftarrow 1$ \KwTo $n$}{
  \For{$j\leftarrow n$ \KwTo $k$}{
  $ g[j]=(g[j]-g[j-1])/(x_j-x_{j-k})$\;
  }
  }
  \tcp{以嵌套乘法形式计算$N_n(x)$}
  $N_n(x)\leftarrow g[n]$\;
  \For{$i\leftarrow n-1$ \KwTo $0$}{
  $N_n(x)\leftarrow (x-x_i)N_n(x)+g[i]$
  }
  \caption{Newton插值}
  \label{algo:newton}
\end{algorithm}

该算法预处理(计算差商)的时间复杂度为$\mathbb{O}(n^2)$,计算某个$x$处的插值函数的时间复杂度为$\mathbb{O}(n)$;空间复杂度为$\mathbb{O}(n)$。

\newpage

\subsection{Neville算法}

记以$x_0,x_1,\dots,x_n$为插值节点构造的$n$次插值多项式为$P_{0,1,\cdots,n}(x)$。存在递归关系

\begin{equation}
  P_{0,1,\cdots,n}(x)=\frac{(x-x_0)P_{1,\cdots,n}(x)-(x-x_n)P_{0,\cdots,n-1}}{x_n-x_0}
\end{equation}

类似于Newton插值中差商表的构造方式,维护数组$p[i],i=0,1,\dots,n$,使得递推结束后,$p[n]=P_{0,\cdots,n}(x)$。完整算法如下:

\begin{algorithm}[h]
  \SetAlgoLined
  \KwData{$(x_i,f(x_i)),i=0,1,\dots,n$;待计算的点$x$。}
  \KwResult{$P_{0,\cdots,n}(x)$}
  
  $p[i]\leftarrow f(x_i), i=0,1,\dots,n$\;
  \For{$i\leftarrow 1$ \KwTo $n$}{
  \For{$j\leftarrow 1$ \KwTo $n-i$}{
  $ p[j]\leftarrow ((x-x_j)p[j+1]-(x-x_{j+i})p[j])/(x_{j+i} - x_j)$\;
  }
  \tcp{$p[n]$即为输出结果$P_{0,\cdots,n}(x)$}
  }
  \caption{Neville算法}
  \label{algo:neville}
\end{algorithm}

显然,该算法的时间复杂度为$\mathbb{O}(n^2)$,空间复杂度为$\mathbb{O}(n)$。相较于Lagrange插值,Neville算法的运算次数更少,是以空间效率换取时间效率的做法。

考虑到插值多项式的唯一性,不再重复估算误差。

\section{结果分析}

由三种多项式插值算法的数值与图示结果可以看出:

\begin{itemize}
  \item 用Lagrange插值估计得1910年的人口数约为31872千人,1965年的人口数约为193082.5千人,2002年的人口数约为26138.75千人。
  \item 利用事后估计方法,计算出1965年的人口数绝对误差约为-1228.4千人,用近似值代替精确值,计算得相对误差约为$-0.64\%$,准确性较好;2002年的人口数绝对误差约为2128310千人,远远超出误差允许范围,准确性极差。
  \item 用Newton插值估计得1965年的人口数约为193082.5千人,2012年的人口数约为-136453千人。
  \item 选取相同的插值点,无论Lagrange插值多项式、Newton插值多项式或用Neville算法得到的插值多项式,均得到相同的数值和图示结果,映证了插值多项式的唯一性。
  \item 在插值区间内,例如1965年,多项式插值的准确性较好;在插值区间外,例如2002、2012年,多项式插值估算得到的值不具有参考价值。
\end{itemize}

\backmatter

\appendix

\chapter{Python程序代码}

\subsection{Lagrange插值}

\pylistings[]{code/lagrange.py}

\newpage

\subsection{Newton插值}

\pylistings[]{code/newton.py}

\newpage

\subsection{Neville算法}

\pylistings[]{code/neville.py}

\newpage

\chapter{CSV格式数据}

\pylistings[]{code/data_points.csv}

\end{document}
